\documentclass[a4paper,10pt]{article}

\usepackage[utf8]{inputenc}
\usepackage[hungarian]{babel}
\usepackage{lmodern}
\usepackage{listings}
\usepackage{wrapfig}
\usepackage{graphicx}

\title{Site generator felhasználói kézikönyv}
\author{Thaler Benedek}
\date{November 2010}
\begin{document}



\maketitle
% insert the table of contents
\tableofcontents

\section{Bevezető}
A program lehetővé teszi HTML fájlok generálását nyers szövegfájlokból, nem igényelve semmilyen HTML jelölőnyelvi ismeretet a felhasználó részéről. Támogat alapvető formázásokat, hivatkozások és képek beszúrását, továbbá megvalósít egy \emph{layout} rendszert, mely lehetővé teszi az oldalak sablonba illesztését.
A program parancssorból futtatható, hibamentes futás esetén nem ír a \emph{standard} kimenetre. Az esetleges problémákat a meghatározott hibakimeneten (\emph{stderr}) jelzi.

\section{Könyvtárstruktúra}
A fájl futtatása során a bemeneten specifikálhatunk egy forrás mappát, mely alapértelmezetten az aktuális könyvtár. 

\begin{lstlisting}[language=bash]
$ site-generator input-directory/
\end{lstlisting}

A weboldal ebben a könyvtárban található fájlok alapján fog elkészülni. Minden forráskönyvtár gyökerében léteznie kell egy index fájlnak, melyben a feldolgozandó fájlokat listázzuk.

A bemeneten specifikálhatjuk a célkönyvtárat is. Ide fogja a program másolni az elkészített HTML fájlokat.

\begin{lstlisting}[language=bash]
$ site-generator input-directory/ output-directory/
\end{lstlisting}

Lehetőség van a generált HTML fájlokba képeket is beilleszteni, melyek külön fájlokat alkotnak. A beillesztendő képeket célszerű a célkönyvtárban tárolni, hogy a böngésző a könyvtár esetleges áthelyezése után is (például célszerverre másolás) meg tudja jeleníteni a beillesztett képeket.
Ha nem szeretnénk összekeverni a generált HTML fájlokat a beillesztendő képekkel, lehetőség van külön képkönyvtár specifikálására, az \texttt{--img} kapcsoló segítségével.

\begin{lstlisting}[language=bash]
$ site-generator input-directory/ output-directory/ --img images/
\end{lstlisting}

A megadott képkönyvtár a kimenetihez képest relatív, tehát a fenti példa feltételezi, hogy az \emph{images} könyvtár az \emph{output-directory} eleme.
Amennyiben nem specifikálunk be- és kimeneti könyvtárat, nincs lehetőség a képkönyvtár meghatározására sem.

\begin{figure}[h]
	\begin{center}
		\includegraphics[scale=0.5]{dirtree}
		\caption{Példa forrás mappa}
	\end{center}
\end{figure}


\section{Fájltípusok}
A bemeneti könyvtár különbböző kiterjesztésű, struktúrűjú fájlokat tartalmazhat. Ezek szervezése egyszerű, funkciójuk logikai alapon különválasztott.

\subsection{Index fájl}
A bemeneti könyvtárnak minden esetben kötelező tartalmaznia egy úgynevezett index fájlt. Ez a fájl teszi lehetővé, hogy a program feldolgozzon minden olyan fájlt, amelyekből a kész weboldalt előállítja.
Az index fájlnak tehát tartalmaznia kell minden szükséges fájl nevét, soronként pontosan egyet.

\begin{wrapfigure}{r}{0.3\textwidth}
		\centering
		\fbox{
			\begin{minipage}[b]{1.1in}

			about-me.widget \\
			footer.widget \\
			layout \\
			about.page \\
			contact.page \\
			works.page \\
			index.page

			\end{minipage}
		}
		\caption{Példa index fájl}
\end{wrapfigure}

Amennyiben a program nem találja az \emph{index} nevű fájlt a bemeneti könyvtár gyökerében, a következő hibaüzenet után kilép;

\begin{lstlisting}
Failed to read index! Aborting...
\end{lstlisting}

Ekkor hozzuk létre az \emph{index} fájlt, vagy ha az már létezik, ellenőrozzük, hogy rendelkezik-e programunk a szükséges olvasási jogokkal.

Amennyiben egy, az index fájlban definiált feldolgozandó fájl megnyitása vagy olvasása során a program akadályba ütközik, a következő hibaüzenet után folytatja a futást;

\begin{lstlisting}
Failed to read file: 'file-name', file skipped.
\end{lstlisting}

Hozzuk létre a hivatkozott fájlt, vagy -- amennyiben nincs szükség rá -- töröljük az indexbejegyzését.

\subsection{Layout fájl}
A layout fájlt definiálja a layout rendszer által haszált keretet, melyet bővebben az \ref{sec:layout}. fejezet mutat be. A layout fájl nem rendelkezik kiterjesztéssel, és gyakori, hogy program által biztosított formázási lehetőségeken túl más HTML jelölőket is használ.

\subsection{Page fájlok}
A .page kiterjesztésű fájlok alkotják a generált weboldal gerincét. Minden egyes ilyen fájl egy külön oldalt reprezentál a kész generátumban. Lehetőség van a szöveg formázására, erről a \ref{sec:format}. fejezetben olvashatunk többet.
Praktikus szempontokból fontos, hogy minden oldalon ismétlődő részeket ne illesszünk be .page fájljainkba, hanem tároljuk őket külön a \emph{layout}ban vagy \emph{widget}ekben, elkerülve így a felesleges kódduplikálást.
\subsection{Widgetek}
Widgetek segítségével oldalunkat kisebb, önálló logikai egységekre tagolhatjuk, mely egységeket a \emph{layout} tetszőleges pontjára beilleszthetünk. Így lehetőség nyílik a gyakrabban használt oldalelemek duplikálásának elkerülésére.
\section{Formázás}
\label{sec:format}
Lehetőségünk van a generált HTML fájlok szövegformázására is, speciális HTML jelölők ismerete nélkül. Használatuk egyszerű; a megfelelő határolókarakterekkel kell körbevenni a szöveget a szövegkiemelés használatához. A \ref{table:textformat}. számú táblázat mutatja be a forrásfájlokban alkalmazható eszközök használatát és hatását.

\begin{table}[h!]
  \begin{center}
    \begin{tabular}{| l | l |}
    \hline
    Forrás szöveg & Eredmény \\
    \hline \hline
    \_dőlt\_ & \textit{dőlt} \\
    \hline
    \*félkövér\* & \textbf{félkövér} \\
    \hline
    \end{tabular}
  \end{center}
  \caption{Szövegformázás}
  \label{table:textformat}
\end{table}

\subsection{Linkek beszúrása}
Szövegünkben használhatunk \emph{link}eket (hiperhivatkozásokat), melyek oldalunk más részeire, vagy külső weboldalakra, internetes forrásokra mutatnak. A linkelni kívánt szöveget egy speciális jelölővel kell körülvenni, melyben meg kell határozni a link célját.

\begin{figure}[h]
	\begin{center}
		[link:about]About me[/l]
		\caption{Belső oldalra mutató link, célja: \emph{about.html}}
		\label{fig:loclink}
	\end{center}
\end{figure}

A ~\ref{fig:loclink}. ábrában látható módon tudunk létrehozni szövegünkben hivatkozásokat. Amennyiben a link célja megegyezik egy általunk -- az index fájlban definiált -- oldal nevével, a hivatkozás lokális lesz; ellenkező esetben pedig külső oldalra fog mutatni.

\subsection{Képek beszúrása}
Természetesen lehetőség van képek beszúrására is. A ~\ref{fig:insimg}. ábrán látható módon illeszthető a megadott útvonalon található \emph{sample.png} nevű kép a szövegbe.

\begin{figure}[h]
	\begin{center}
		[img:sample.png]
		\caption{Kép beszúrása}
		\label{fig:insimg}
	\end{center}
\end{figure}

Célszerű a képeinket a kimeneti könyvtárban tárolni, mivel azokat nem másolja át a program a feldolgozás során. Így biztosak lehetünk abban, hogy a kész HTML oldalakon megjelenik majd a beillesztett kép.

\section{Layout rendszer}
\label{sec:layout}
A program által megvalósított \emph{layout} rendszer teszi lehetővé, hogy oldalaink számára egyszerre egységes és könnyen módosítható megjelenést biztosítsunk, tartalomduplikáció nélkül. A \emph{layout} egy keretet biztosít, mely minden oldalt körülvesz; gondolhatunk rá úgy, mint egy alaprajzra, mely meghatározza oldalunk főb elemeinek az elrendezését. Használatához hozzunk létre egy \emph{layout} nevű fájlt a forrásmappában, és adjuk hozzá az \emph{index} fájl listájához egy új sorban.
A program a feldolgozás során minden egyes oldal tartalmát beilleszti a \emph{layout}ban meghatározott helyre. Ezt a helyet egy speciális \emph{helyörző}vel tudjuk meghatározni, az ~\ref{fig:layout}. ábrán látható módon.

\begin{figure}[h]
	\begin{center}
		\begin{lstlisting}
<html>
	<head><title>My portfolio</title></head>
	<body>
		{content}
	</body>
</html>
		\end{lstlisting}
		\caption{A \emph{layout} használata}
		\label{fig:layout}
	\end{center}
\end{figure}

A példában a HTML oldal törzsébe, a \texttt{\{content\}} helyörző helyére fog bekerülni minden egyes oldal, külön-külön.

\subsection{Navigáció}
Oldalunk navigációját nehéz és körülményes lenne konzisztensen tartani, ha minden alkalommal, amikor új oldalt hozunk létre, módosítanunk kéne a navigációt is. A probléma megoldására a \emph{layout} rendszer nyújt megoldást: helyezzük el a \texttt{\{navigation\}} helyörzőt a \emph{layout} fájlunk tetszőleges pontjára. Programunk minden egyes oldal esetében egy menürendszert fog generálni, melynek elemei weboldalunk különböző oldalaira mutatnak.

\subsection{Widgetek beillesztése}
A \emph{widget}ek teszik lehetővé, hogy oldalunk ismétlődő részeit logikai egységekre bontva, külön tároljuk. Használatukhoz hozzunk létre egy tetszőleges nevű fájlt a forráskönyvtárban, \texttt{.widget} kiterjesztéssel, és a létrehozott fájlt adjuk hozzá \emph{index} fájlunkhoz egy új sorban. Az újonnan létrehozott \emph{widget}ünkre a fájl nevével tudunk hivatkozni. A ~\ref{fig:widget}. ábra mutatja be egy egyszerű widget használatát.

\begin{figure}[h]
	\begin{center}
		\begin{lstlisting}
<html>
	<head><title>My portfolio</title></head>
	<body>
		{content}
	</body>
	<side>
		{widget:about-me.widget}
	</side>
</html>
		\end{lstlisting}
		\caption{Egy saját widget használata}
		\label{fig:widget}
	\end{center}
\end{figure}



\end{document}
