\documentclass[a4paper,10pt]{article}

\usepackage[utf8]{inputenc}
\usepackage[hungarian]{babel}
\usepackage{lmodern}

\title{Site generator felhasználói kézikönyv}
\author{Thaler Benedek}
\date{November 2010}
\begin{document}



\maketitle
% insert the table of contents
\tableofcontents

\section{Bevezető}

\section{Könyvtárstruktúra}
A fájl futtatása során a bemeneten specifikálhatunk egy forrás mappát, mely alapértelmezetten az aktuális könyvtár. 

\texttt{> site-generator input-directory/}

A weboldal ebben a könyvtárban található fájlok alapján fog elkészülni. Minden forráskönyvtár gyökerében léteznie kell egy index fájlnak, melyben a feldolgozandó fájlokat listázzuk.

A bemeneten specifikálhatjuk a célkönyvtárat is. Ide fogja a program másolni az elkészített HTML fájlokat.

\texttt{> site-generator input-directory/ output-directory/}

Lehetőség van a generált HTML fájlokba képeket is beilleszteni, melyek külön fájlokat alkotnak. A beillesztendő képeket célszerű a célkönyvtárban tárolni, hogy a böngésző a könyvtár esetleges áthelyezése után is (például célszerverre másolás) meg tudja jeleníteni a beillesztett képeket.
Ha nem szeretnénk összekeverni a generált HTML fájlokat a beillesztendő képekkel, lehetőség van külön képkönyvtár specifikálására, az \texttt{--img} kapcsoló segítségével.

\texttt{> site-generator input-directory/ output-directory/ --img images/}

A megadott képkönyvtár a kimenetihez képest relatív, tehát a fenti példa feltételezi, hogy az \emph{images} könyvtár az \emph{output-directory} eleme.
Amennyiben nem specifikálunk be- és kimeneti könyvtárat, nincs lehetőség a képkönyvtár meghatározására sem.

%figure: könyvtárstruktúra%


\section{Fájltípusok} %subsections: index, layout, page, widget
A bemeneti könyvtár különbböző kiterjesztésű, struktúrűjú fájlokat tartalmazhat. Ezek szervezése egyszerű, funkciójuk logikai alapon különválasztott.

\subsection{Index fájl}
A bemeneti könyvtárnak minden esetben kötelező tartalmaznia egy úgynevezett index fájlt. Ez a fájl teszi lehetővé, hogy a program feldolgozzon minden olyan fájlt, amelyekből a kész weboldalt előállítja.
Az index fájlnak tehát tartalmaznia kell minden szükséges fájl nevét, soronként pontosan egyet.

%example
about-me.widget \\
footer.widget \\
about.page \\
contact.page \\
works.page \\
%example

Amennyiben a program nem találja az \emph{index} nevű fájlt a bemeneti könyvtár gyökerében, a következő hibaüzenet után kilép;

\texttt{Failed to read index! Aborting...}

Ekkor hozzuk létre az \emph{index} fájlt, vagy ha az már létezik, ellenőrozzük, hogy rendelkezik-e programunk a szükséges olvasási jogokkal.

Amennyiben egy, az index fájlban definiált feldolgozandó fájl megnyitása vagy olvasása során a program akadályba ütközik, a következő hibaüzenet után folytatja a futást;

\texttt{Failed to read file: 'file-name', file skipped.}

Hozzuk létre a hiányolt fájlt, vagy - amennyiben nincs szükség rá - töröljük az indexbejegyzését.

\subsection{Layout fájl}
\subsection{Page fájlok}
A .page kiterjesztésű fájlok alkotják a generált weboldal gerincét. Minden egyes ilyen fájl egy külön oldalt reprezentál a kész generátumban. Lehetőség van a szöveg formázására, erről a \ref{sec:format}. fejezetben olvashatunk többet.
Praktikus szempontokból fontos, hogy minden oldalon ismétlődő részeket ne illesszünk be .page fájljainkba, hanem tároljuk őket külön a \emph{layout}ban vagy \emph{widget}ekben, elkerülve így a felesleges kódduplikálást.
\subsection{widget}
Widgetek segítségével oldalunkat kisebb, önálló logikai egységekre tagolhatjuk, mely egységeket a \emph{layout} tetszőleges pontjára beilleszthetünk. Így lehetőség nyílik a gyakrabban használt oldalelemek duplikálásának elkerülésére.
\section{Formázás}
\label{sec:format}
\section{Layout rendszer}
\subsection{Navigáció}
\subsection{Widgetek beillesztése}

\end{document}
